% Created 2011-05-20 Fri 11:21
\documentclass[10pt,english]{article}
\usepackage[utf8]{inputenc}
\usepackage[T1]{fontenc}
\usepackage{fixltx2e}
\usepackage{graphicx}
\usepackage{longtable}
\usepackage{float}
\usepackage{wrapfig}
\usepackage{soul}
\usepackage{textcomp}
\usepackage{marvosym}
\usepackage{wasysym}
\usepackage{latexsym}
\usepackage{amssymb}
\usepackage{hyperref}
\tolerance=1000

\usepackage{lmodern}
\renewcommand{\sfdefault}{lmss}
\renewcommand{\ttdefault}{lmtt}

% needed packages
\usepackage{amsmath}
\usepackage{amssymb}
\usepackage{amsthm}
\usepackage{babel}
\usepackage{epsfig}
\usepackage[T1]{fontenc}
\usepackage{fixltx2e}
\usepackage{float}
%\usepackage{floatflt}
\usepackage{graphics}
\usepackage{graphicx}
\usepackage[utf8]{inputenc}
\usepackage{latexsym}
\usepackage{longtable}
\usepackage{makeidx}
\usepackage{marvosym}
\usepackage{multicol}
%\usepackage{pslatex}
\usepackage{rotating}
%\usepackage{showidx}
\usepackage{soul}
\usepackage{srcltx}
\usepackage{stmaryrd}
\usepackage{subfig}
\usepackage{textcomp}
%\usepackage{theorem}
%\usepackage[subfigure]{tocloft}
\usepackage{txfonts}
\usepackage{upgreek}
\usepackage{url}
\usepackage{varioref}
%\usepackage{wasysym}
\usepackage{wrapfig}


% Page setup
\usepackage[paperwidth=8.5in,paperheight=11in]{geometry}
\geometry{verbose,tmargin=0.5in,bmargin=0.5in,lmargin=1in,rmargin=1in}




% PDF settings
%\usepackage[hyperref,x11names]{xcolor}
\usepackage{hyperref}
\hypersetup{pdftitle={STAT 5840: Statistical Computing},
 		pdfauthor={G. Jay Kerns}, 
		linkcolor=Firebrick4, 
		citecolor=black, 
		urlcolor=SteelBlue4}

% Listings setup
%\usepackage{color}
%\usepackage{listings}
%\lstset{basicstyle={\ttfamily},
%	language=R,
%	breaklines=true,
%	breakatwhitespace=true,
%	keywordstyle={\ttfamily},
%	numberstyle = {\ttfamily},
%	morestring=[b]"
%}



%  user defined commands
% special operators
\renewcommand{\P}{\mathrm{I\hspace{-1.5pt}P}}
\newcommand{\E}{\mathrm{I\hspace{-1.5pt}E}}
\renewcommand{\vec}[1]{\mbox{\boldmath$#1$}}

% special symbols
\newcommand{\me}{\mathrm{e}}
\newcommand{\R}{\mathbb{R}}
\newcommand{\diff}{\mathrm{d}}
\newcommand{\ybar}{\overline{y}}
\newcommand{\xbar}{\overline{x}}
\newcommand{\Xbar}{\overline{X}}
\newcommand{\Ybar}{\overline{Y}}





\providecommand{\alert}[1]{\textbf{#1}}

\title{HOWTO Submit Assignments for STAT 5840}
%\author{G. Jay Kerns}
\date{STAT 5840: Summer 2011}

\begin{document}

\maketitle

\thispagestyle{empty}


\section*{Who}
\label{sec-1}

You.
\section*{What}
\label{sec-2}

All STAT 5840 assignments to be turned in for a grade should be submitted as \texttt{Sweave} files, that is, a text file with file extension \texttt{.Rnw} which contains \texttt{R} code mixed with \(\LaTeX\) code to ultimately come together to comprise a self-contained statistical report. Your file should be named according to the following scheme:

\begin{verbatim}
 LastNameDueDate.Rnw
\end{verbatim}


So, for example, if I were going to submit an assignment due on May 27, I would submit a file called \texttt{Kerns052711.Rnw}.
\section*{When}
\label{sec-3}

Assignments are due \emph{before class} on the due-date assigned.  If it isn't in my email INBOX by 1:00pm, then it's late, period.  Early submissions (to protect against the occasional email snafu) are welcomed and encouraged (for the final exam assignment, in particular).
\section*{Where}
\label{sec-4}

Send STAT 5840 submissions to my YSU email address, \texttt{gkerns@ysu.edu}.  Attach your \texttt{.Rnw} file to the email.  The subject line of the email should read \texttt{STAT 5840 LastName DueDate}.  So, for instance, if I were turning in an assignment due on May 27 my email subject would read \texttt{STAT 5840 Kerns 052711}.
\section*{How}
\label{sec-5}

Use the templates I provide plus the \texttt{R} scripts we discuss in class to help you write your report(s).  In every case the work you need to do is scantly more than copy-pasting the work I've done and tweaking it a little bit for your particular assignment.  Don't reinvent the wheel;  it's rolling right there in front of you.  If you have a personal computer at home and can manage to get it up and running to meet all of the installation requirements, then I encourage you to work at home.  Everything's \textbf{free}, after all.  Nevertheless, if you \emph{don't} own a personal computer or can't manage to get everything running, then you can always use the Computer Lab in Lincoln 414.  I have checked it out and confirmed that everything there works without any trouble. 
\section*{Why}
\label{sec-6}


\begin{description}
\item[Short answer:] you'd like to pass the class.
\item[Long answer:] you have a burning, aching desire in the deepest reachest of your fiery intellect to become the baddest-assed computational statistician this side of the Mississippi River, and submission of a data analysis assignment for STAT 5840 Summer 2011 is, in your view, one of the preliminary first steps in a long series of incremental successes toward that ultimate goal of superlative awesome-ness.
\end{description}

\end{document}