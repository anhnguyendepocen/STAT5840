\documentclass[english]{article}
\usepackage[T1]{fontenc}
\usepackage[latin9]{inputenc}
\usepackage{geometry}
\geometry{verbose,tmargin=0.5in,lmargin=0.9in}
\usepackage{color}
\usepackage{url}
\usepackage{amssymb}

\makeatletter
%%%%%%%%%%%%%%%%%%%%%%%%%%%%%% Textclass specific LaTeX commands.
\usepackage{Sweave}
\newcommand{\Rcode}[1]{{\texttt{#1}}}
\newcommand{\Robject}[1]{{\texttt{#1}}}
\newcommand{\Rcommand}[1]{{\texttt{#1}}}
\newcommand{\Rfunction}[1]{{\texttt{#1}}}
\newcommand{\Rfunarg}[1]{{\textit{#1}}}
\newcommand{\Rpackage}[1]{{\textit{#1}}}
\newcommand{\Rmethod}[1]{{\textit{#1}}}
\newcommand{\Rclass}[1]{{\textit{#1}}}

%%%%%%%%%%%%%%%%%%%%%%%%%%%%%% User specified LaTeX commands.

% packages used
\usepackage{amssymb,latexsym}
\usepackage[mathscr]{eucal}
\usepackage{amsthm}
\usepackage{amsmath}
\usepackage{amsfonts}
\usepackage{layout}
\usepackage{color}
\usepackage{multicol}
\usepackage{hyperref}

% get rid of extra Sweave space
\fvset{listparameters={\setlength{\topsep}{0pt}}}
\renewenvironment{Schunk}{\vspace{\topsep}}{\vspace{\topsep}}  

\makeatother

\usepackage{babel}
\begin{document}

\title{A \emph{Brief} Introduction to \textsf{R}}

\maketitle
This is a document designed to help a person to begin to get to know
the \textsf{R} statistical computing environment. It paraphrases and
summarizes information gleaned from materials listed in the \textbf{References}.
Please refer to them for a more complete treatment.


\section{Installing \textsf{R} and the \texttt{IPSUR} Package}

There are detailed instructions for installing \textsf{R} on your
personal computer at the following website:

\begin{center}
\url{http://ipsur.r-forge.r-project.org/book/installation.php}
\par\end{center}

\noindent For more complete and technical installation instructions
see the \emph{R Installation and Administration Manual}.

\begin{center}
\url{http://cran.r-project.org/doc/manuals/R-admin.html}
\par\end{center}

\noindent You will also want to install RStudio, which you can find
here:

\begin{center}
\url{http://www.rstudio.org/}
\par\end{center}


\section{Communicating with \textsf{R}}

There are three basic methods for communicating with the software.
\begin{enumerate}
\item At the Command Prompt (\texttt{>}).


This is the most basic way to complete simple, one-line commands.
\textsf{R} will evaluate what is typed there and output the results
in the Console Window.

\item Copy \& Paste from a text file.


For longer programs (called \emph{scripts}) there is too much code
to write all at once at the Command Prompt. Further, for long scripts
the user sometimes wishes to only modify a certain piece of the script
and run it again in \textsf{R}. 


One way to do this is to open a text file with a text editor (say,
NotePad or MS-Word$\circledR$). One writes the code in the text file,
then when satisfied the user copy-and-pastes it at the Command Prompt
in \textsf{R}. Then \textsf{R} will compile all of the code at once
and give output in the Console Window. 


Alternatively, \textsf{R} provides its own built-in script editor,
called \textsf{R} Editor. From the console window, select \emph{File
$\rightarrow$ New} \emph{Script.} A script window opens, and the
lines of code can be written in the window. When satisfied with the
code, the user highlights all of the commands and presses \texttt{Ctrl+R}.
The commands are automatically run at once in \textsf{R} and the output
is shown. To save the script for later, click \emph{File $\rightarrow$
Save as...} in \textsf{R} Editor. The script can be reopened later
with \emph{File $\rightarrow$ Open Script...} in the Console Window.


A disadvantage to these methods is that all of the code is written
in the same way, with the same font. It can become confusing with
longer scripts, and there is no way to efficiently identify mistakes
in the code. To address this problem, software developers have designed
powerful IDE / Script Editors.

\item IDE / Script Editors.


There are free programs specially designed to aid the communication
and code writing process. The advantage to using Script Editors is
that they have additional functions and options to help the user write
code more efficiently, including \textsf{R} syntax highlighting, automatic
code completion, delimiter matching, and dynamic help on the \textsf{R}
functions as they are written. In addition, they typically have all
of the text editing features of programs like MS Word. Lastly, most
script editors are fully customizable in the sense that the user can
customize the appearance of the interface and can choose what colors
to display, when to display them, and how they are to be displayed.


Some of the more popular script editors can be downloaded from the
R-Project website at \\
\url{http://www.sciviews.org/_rgui/}. On the left side of the screen
(under \textbf{Projects}) there are several choices available.

\begin{itemize}
\item \textbf{RStudio}: This open-source IDE works on all platforms and
is the best one for beginners that I have yet seen to date.
\item \textbf{Tinn-}\textsf{\textbf{R}}: This one is also free, but is really
just used for editing scripts.
\item \textbf{Emacs} / \textbf{ESS}: Click Emacs (ESS) or Emacs (ESS/Windows).
This will take you to download sites with sophisticated programs for
editing, compiling, and coordinating software such as \texttt{S-Plus},
\textsf{R}, and \texttt{SAS} simultaneously. Emacs is short for \emph{E}diting
\emph{MAC}ro\emph{S} and ESS means \emph{E}macs \emph{S}peaks \emph{S}tatistics.
An alternate branch of Emacs is called XEmacs. This editor is -- \emph{by
far} -- the most powerful of the text editors, but all of the flexibility
comes at a price. Emacs requires a level of computer-saavy that the
others do not, and the learning curve is more steep. If you want to
explore this option, then speak with me beforehand; I can give you
some advice about getting started.
\end{itemize}

\end{enumerate}

\section{A First Session: Using \textsf{R} as a calculator}


\textsf{R} is perfectly able to do standard calculations. For example,
type $2+3$ and observe

\begin{Schunk}
\begin{Sinput}
> 2 + 3
\end{Sinput}
\begin{Soutput}
[1] 5
\end{Soutput}
\end{Schunk}

The \texttt{{[}1{]}} means that the 5 is the first entry in the list,
and the \texttt{>} means that \textsf{R} is waiting on your next command.
Entry numbers will be generated for each row, such as

\begin{Schunk}
\begin{Sinput}
> 3:50
\end{Sinput}
\begin{Soutput}
 [1]  3  4  5  6  7  8  9 10 11 12 13 14 15 16 17 18 19 20 21 22 23 24 25 26 27
[26] 28 29 30 31 32 33 34 35 36 37 38 39 40 41 42 43 44 45 46 47 48 49 50
\end{Soutput}
\end{Schunk}


Here, the $26^{\mathrm{th}}$ entry in the list is 28. Notice also
the \texttt{3:50} notation, which generates all numbers in sequence
from 3 to 50. One can also do things like

\begin{Schunk}
\begin{Sinput}
> 2 * 3 * 4 * 5  # multiply
\end{Sinput}
\begin{Soutput}
[1] 120
\end{Soutput}
\begin{Sinput}
> sqrt(10)       # square root
\end{Sinput}
\begin{Soutput}
[1] 3.162278
\end{Soutput}
\begin{Sinput}
> pi             # pi
\end{Sinput}
\begin{Soutput}
[1] 3.141593
\end{Soutput}
\begin{Sinput}
> sqrt(-2)
\end{Sinput}
\begin{Soutput}
[1] NaN
\end{Soutput}
\end{Schunk}


Notice that \texttt{NaNs} were produced; this stands for ``not
a number''. Also notice the number sign \texttt{\#}, which means
\emph{comment}. Everything typed on the same line after the \texttt{\#}
will be ignored by \textsf{R}. There is also a continuation prompt
\texttt{+} which occurs if you press \texttt{Enter} before a statement
is complete. For example, if you forget to close the parentheses or
a command you may get something like the following:

\begin{Schunk}
\begin{Sinput}
> sqrt(27 + 32
+
+
\end{Sinput}
\end{Schunk}

To exit out of the continuation prompt, you can either complete the
command - by entering a \texttt{)} in the above example
- or you may press the \texttt{Esc} key.

Some other fuctions that will be of use are \texttt{abs()} for absolute
value, \texttt{log()} for the natural logarithm, \texttt{exp()} for
the exponential function, \texttt{factorial()} for computing permutations,
and \texttt{choose()} for binomial coefficients.

\textbf{Assignment.} This is useful for storing values to be used
later.

\begin{Schunk}
\begin{Sinput}
> y = 5    # stores the value 5 in y
> y
\end{Sinput}
\begin{Soutput}
[1] 5
\end{Soutput}
\begin{Sinput}
> y <- 5   # also stores the value 5 in y
> 7 -> z   # stores the value 7 in z
\end{Sinput}
\end{Schunk}

You do not have to use the \texttt{<-} notation to store things; the
equal sign \texttt{=} works just as well. I will use both symbols
interchangeably.

\textbf{Acceptable variable names.} You can use letters, numbers,
dots ``\texttt{.}'', or underscore ``\texttt{\_}'' characters.
You cannot use mathematical operators, and a leading dot may not be
followed by a number. Examples: \texttt{x}, \texttt{x1}, \texttt{y32},
\texttt{x.variable}, \texttt{x\_variable}.

\textbf{Using} \texttt{\textbf{c()}} \textbf{to enter data vectors.}
If you would like to enter the data \texttt{74,31,95,61,76,34,23,54,96}
into \textsf{R}, you may create a data vector with the \texttt{c()}
function (short for \emph{concatenate}). 

\begin{Schunk}
\begin{Sinput}
> fred = c(74, 31, 95, 61, 76, 34, 23, 54, 96)
> fred
\end{Sinput}
\begin{Soutput}
[1] 74 31 95 61 76 34 23 54 96
\end{Soutput}
\end{Schunk}

The vector \texttt{fred} has 9 entries. We can access individual components
with bracket \texttt{[]} notation:

\begin{Schunk}
\begin{Sinput}
> fred[3]
\end{Sinput}
\begin{Soutput}
[1] 95
\end{Soutput}
\begin{Sinput}
> fred[2:4]
\end{Sinput}
\begin{Soutput}
[1] 31 95 61
\end{Soutput}
\begin{Sinput}
> fred[c(1, 3, 5, 8)]
\end{Sinput}
\begin{Soutput}
[1] 74 95 76 54
\end{Soutput}
\end{Schunk}


If you would like to reset the variable \texttt{fred}, you can do
it by typing \texttt{fred = c()}.

\textbf{Using} \texttt{\textbf{scan()}} \textbf{to enter numeric data
vectors}. If you would like to enter the data 76 34 23 54 96 into
a vector \texttt{x}, perhaps the quickest way would be to use the
\texttt{scan()} function:

\begin{Schunk}
\begin{Sinput}
> x <- scan()
1: 76
2: 34
3: 23
4: 54
5: 9
66: 
\end{Sinput}
\begin{Soutput}
Read 5 items
\end{Soutput}
\end{Schunk}

This method is best suited for use with small data sets and \textbf{only
works if the data are numeric}. Notice that entering an empty line
stops the scan. Another use of this feature is when you have a long
list of numbers (separated by spaces or on different lines) already
typed somewhere else, say in a text file. To enter all the data in
one fell swoop, highlight and copy the list of numbers to the Clipboard
with \emph{Edit $\rightarrow$ Copy}, next type the \texttt{x = scan()}
command in the \textsf{R} console, and paste the numbers at the \texttt{1:}
prompt with \emph{Edit $\rightarrow$ Paste}. All of the numbers will
automatically be entered into the vector \texttt{x}.

\textbf{Data vectors have type}. There are numeric, character, and
logical type vectors. If you mix and match then usually it will be
character. Notice that characters can be identified with either single
or double quotes.

\begin{Schunk}
\begin{Sinput}
> simpsons = c("Homer", "Marge", "Bart", "Lisa", "Maggie")
> names(simpsons) = c("dad", "mom", "son", "daughter 1", "daughter 2")
> simpsons
\end{Sinput}
\begin{Soutput}
       dad        mom        son daughter 1 daughter 2 
   "Homer"    "Marge"     "Bart"     "Lisa"   "Maggie" 
\end{Soutput}
\end{Schunk}


Here is an example of a logical vector:

\begin{Schunk}
\begin{Sinput}
> x = c(5, 7)
> v = (x < 6)
\end{Sinput}
\end{Schunk}


\textbf{Applying functions to a data vector.} Once we have stored
a data vector then we can evaluate functions on it. 

\begin{Schunk}
\begin{Sinput}
> fred
\end{Sinput}
\begin{Soutput}
[1] 74 31 95 61 76 34 23 54 96
\end{Soutput}
\begin{Sinput}
> sum(fred)
\end{Sinput}
\begin{Soutput}
[1] 544
\end{Soutput}
\begin{Sinput}
> length(fred)
\end{Sinput}
\begin{Soutput}
[1] 9
\end{Soutput}
\begin{Sinput}
> sum(fred)/length(fred)
\end{Sinput}
\begin{Soutput}
[1] 60.44444
\end{Soutput}
\begin{Sinput}
> mean(fred)                 # sample mean, should be same answer
\end{Sinput}
\begin{Soutput}
[1] 60.44444
\end{Soutput}
\end{Schunk}


Other popular functions for vectors are \texttt{range()}, \texttt{min()},
\texttt{max()}, \texttt{sort()}, and \texttt{cumsum()}.

\textbf{Vectorizing functions.} Arithmetic in \textsf{R} is almost
always done element-wise, also known as \emph{vectorizing functions}.
Some examples follow.

\begin{Schunk}
\begin{Sinput}
> fred.2 = c(4, 5, 3, 6, 4, 6, 7, 3, 1)
> fred + fred.2
\end{Sinput}
\begin{Soutput}
[1] 78 36 98 67 80 40 30 57 97
\end{Soutput}
\begin{Sinput}
> fred - fred.2
\end{Sinput}
\begin{Soutput}
[1] 70 26 92 55 72 28 16 51 95
\end{Soutput}
\begin{Sinput}
> fred - mean(fred)
\end{Sinput}
\begin{Soutput}
[1]  13.5555556 -29.4444444  34.5555556   0.5555556  15.5555556 -26.4444444
[7] -37.4444444  -6.4444444  35.5555556
\end{Soutput}
\end{Schunk}


The operations \texttt{+} and \texttt{-} are performed element-wise.
Notice in the last vector that \texttt{mean(fred)} was subtracted
from each entry, in turn. This is also known as \emph{data recycling}.
Other popular vectorizing functions are \texttt{sin()}, \texttt{cos()},
\texttt{exp()}, \texttt{log()}, and \texttt{sqrt()}.


\section{Getting Help}

When you are using \textsf{R}, it will not take long before you find
yourself needing help. Fortunately, \textsf{R} has extensive help
resources and you should immediately become familiar with them. Begin
by clicking \emph{Help} on the console. The following options are
available.

\begin{itemize}
\item \textbf{Console}: gives useful shortcuts, for instance, \texttt{Ctrl+L},
to clear the \textsf{R} console screen.
\item \textbf{FAQ on }\textsf{\textbf{R}}: frequently asked questions concerning
general \textsf{R} operation.
\item \textbf{FAQ on }\textsf{\textbf{R}}\textbf{ for Windows}: frequently
asked questions about \textsf{R}, tailored to the Windows operating
system.
\item \textbf{Manuals}: technical manuals about all features of the \textsf{R}
system including installation, the complete language definition, and
add-on packages.
\item \textsf{\textbf{R}}\textbf{ functions (text)\ldots{}}: use this if
you know the \emph{exact} name of the function you want to know more
about, for example, \texttt{mean} or \texttt{plot.} Typing mean in
the window is equivalent to typing \texttt{help(\textquotedbl{}mean\textquotedbl{})}
at the command line, or more simply, \texttt{?mean}.
\item \textbf{Html Help}: use this to browse the manuals with point-and-click
links. It also has a Search Engine \& Keywords for searching the help
page titles, with point-and-click links for the search results. This
is possibly the best help method for beginners.
\item \textbf{Search help\ldots{}}: use this if you do not know the exact
name of the function of interest. For example, you may enter \texttt{plo}
and a text window will return listing help files with an alias, concept,
or title matching 'plo' using regular expression matching; it is equivalent
to typing \texttt{help.search(\textquotedbl{}plo\textquotedbl{})}
at the command line. The advantage is that you do not need to know
the exact name of the function; the disadvantage is that you cannot
point-and-click the results. Therefore, one may wish to use the Html
Help search engine instead.
\item \textbf{search.r-project.org\ldots{}}: this will search for words
in help lists and archives of the \textsf{R} Project. It can be very
useful for finding other questions that useRs have asked.
\item \textbf{Apropos\ldots{}}: use this for more sophisticated partial
name matching of functions. Try \texttt{?apropos} for details.
\end{itemize}
Note also \texttt{example()}. This initiates the running of examples,
if available, of the use of the function specified by the argument.


\section{Other tips}

It is unnecessary to retype commands repeatedly, since \textsf{R}
remembers what you have entered on the command line. To cycle through
the previous commands, just push the $\uparrow$ (up arrow) key. 

Missing values in \textsf{R} are denoted by \texttt{NA}. Operations
on data vector \texttt{NA} values treat them as if the values can't
be found. This means adding (as well as subtracting and all of the
other mathematical operations) a number to \texttt{NA} results in
\texttt{NA}.

To find out what all variables are in the current work environment,
use the commands \texttt{ls()} or \texttt{objects()}. These list all
available objects in the workspace. If you wish to remove one or more
variables, use \texttt{remove(var1, var2)}, and to remove all of them
use \texttt{rm(list=ls())}. 


\section{Some References}
\begin{itemize}
\item Dalgaard, P. (2002). \emph{Introductory Statistics with} \textsf{R}.
Springer.
\item Everitt, B. (2005). \emph{An} \textsf{R} \emph{and} \texttt{S-Plus}
\emph{Companion to Multivariate Analysis}. Springer.
\item Heiberger, R. and Holland, B. (2004). \emph{Statistical Analysis and
Data Display. An Intermediate Course with Examples in} \texttt{S-Plus},
\textsf{R}, \emph{and} \texttt{SAS}. Springer.
\item Maindonald, J. and Braun, J. (2003). \emph{Data Analysis and Graphics
Using} \textsf{R}: \emph{an Example Based Approach}. Cambridge University
Press.
\item Venables, W. and Smith, D. (2005). \emph{An Introduction to} \textsf{R}.
\url{http://www.r-project.org/Manuals}.
\item Verzani, J. (2005). \emph{Using} \textsf{R} for \emph{Introductory
Statistics}. Chapman and Hall.\end{itemize}

\end{document}
