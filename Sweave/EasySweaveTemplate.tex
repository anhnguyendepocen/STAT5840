\documentclass[12pt]{article}         % the type of document and font size (default 10pt)
\usepackage[margin=1.0in]{geometry}   % sets all margins to 1in -- can be changed
\usepackage{moreverb}                 % for verbatimtabinput -- LaTeX environment
\usepackage{url}                      % for \url{} command
\usepackage{amssymb}                  % for many mathematical symbols
\usepackage[pdftex]{lscape}           % for landscaped tables
\usepackage{longtable}                % for tables that break over multiple pages
\title{Easiest Sweave Template Ever}  % to specify title
\author{Your name goes here}          % to specify author(s)
\usepackage{Sweave}
\begin{document}                      % document begins here
\input{EasySweaveTemplate-concordance}

% If .nw file contains graphs: To specify that EPS/PDF graph files are to be 
% saved to 'graphics' sub-folder
%     NOTE: 'graphics' sub-folder must exist prior to Sweave step
%\SweaveOpts{prefix.string=graphics/plot}

% If .nw file contains graphs: to modify (shrink/enlarge} size of graphics 
% file inserted
%         NOTE: can be specified/modified before any graph chunk
\setkeys{Gin}{width=1.0\textwidth}

\maketitle              % makes the title
%\tableofcontents        % inserts TOC (section, sub-section, etc numbers and titles)
%\listoftables           % inserts LOT (numbers and captions)
%\listoffigures          % inserts LOF (numbers and captions)
%                        %     NOTE: graph chunk must be wrapped with \begin{figure}, 
%                        %  \end{figure}, and \caption{}
%%%%%%%%%%%%%%%%%%%%%%%%%%%%%%%%%%%%%%%%%%%%%%%%%%%%%%%%%%%%%%%%%%%%
% Where everything else goes

\section{How to typeset \textsf{R} code}

If you want to see both the input and output, do this:

\begin{Schunk}
\begin{Sinput}
> runif(10)
\end{Sinput}
\begin{Soutput}
 [1] 0.88391527 0.92538512 0.17091107 0.77188165 0.48213965 0.09386202
 [7] 0.06434053 0.04402382 0.87761762 0.84074091
\end{Soutput}
\end{Schunk}

If you want to see output, but no input, do this:

\begin{Schunk}
\begin{Soutput}
 [1] 0.59317417 0.71766509 0.42739176 0.98193059 0.05759842 0.63969684
 [7] 0.79435661 0.61805005 0.86995578 0.30739595
\end{Soutput}
\end{Schunk}

If you want to see input, but no output, do this:

\begin{Schunk}
\begin{Sinput}
> runif(13)
\end{Sinput}
\end{Schunk}

If you want to run some \textsf{R} code but hide the input/output from the reader then you can do both at the same time:


\bigskip   % leave some empty space (optional)

and you can double-check that it worked later (if you like)

\begin{Schunk}
\begin{Sinput}
> x  # use keep.source=TRUE if you want comments printed
\end{Sinput}
\begin{Soutput}
 [1]  2  3  4  5  6  7  8  9 10 11
\end{Soutput}
\begin{Sinput}
> y
\end{Sinput}
\begin{Soutput}
 [1] 0.4818667 0.4489951 0.4414288 0.7814628 0.2379165 0.9063004 0.7147336
 [8] 0.6502742 0.1150843 0.5748972
\end{Soutput}
\end{Schunk}

If you want to write some \textsf{R} code but not have it evaluated at all then do this:

\begin{Schunk}
\begin{Sinput}
> # whatever you write here must be syntactically correct R code
> runif(1000000000000000000000000)
\end{Sinput}
\end{Schunk}

If you would like to include a figure that's generated completely by \textsf{R} code, then you can do something like the following.

\begin{figure}
\includegraphics{EasySweaveTemplate-007}
\caption{Here is the plot we made}
\end{figure}


Sometimes we would like the output to look like \LaTeX\ output instead of \textsf{R} output.  In that case, do the following.

\begin{Schunk}
\begin{Sinput}
> library(xtable)
> xtable(summary(lm(y ~ x)), caption = "Here is the table we made")
\end{Sinput}
% latex table generated in R 2.15.1 by xtable 1.7-0 package
% Mon Oct 15 15:55:36 2012
\begin{table}[ht]
\begin{center}
\begin{tabular}{rrrrr}
  \hline
 & Estimate & Std. Error & t value & Pr($>$$|$t$|$) \\ 
  \hline
(Intercept) & 0.5348 & 0.2013 & 2.66 & 0.0289 \\ 
  x & 0.0001 & 0.0283 & 0.00 & 0.9980 \\ 
   \hline
\end{tabular}
\caption{Here is the table we made}
\end{center}
\end{table}\end{Schunk}


\end{document}


