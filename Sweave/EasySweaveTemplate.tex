\documentclass[12pt]{article}         % the type of document and font size (default 10pt)
\usepackage[margin=1.0in]{geometry}   % sets all margins to 1in -- can be changed
\usepackage{moreverb}                 % for verbatimtabinput -- LaTeX environment
\usepackage{url}                      % for \url{} command
\usepackage{amssymb}                  % for many mathematical symbols
\usepackage[pdftex]{lscape}           % for landscaped tables
\usepackage{longtable}                % for tables that break over multiple pages
\title{Easiest Sweave Template Ever}  % to specify title
\author{Your name goes here}          % to specify author(s)
\usepackage{Sweave}
\begin{document}                      % document begins here

% If .nw file contains graphs: To specify that EPS/PDF graph files are to be 
% saved to 'graphics' sub-folder
%     NOTE: 'graphics' sub-folder must exist prior to Sweave step
%\SweaveOpts{prefix.string=graphics/plot}

% If .nw file contains graphs: to modify (shrink/enlarge} size of graphics 
% file inserted
%         NOTE: can be specified/modified before any graph chunk
\setkeys{Gin}{width=1.0\textwidth}

\maketitle              % makes the title
%\tableofcontents        % inserts TOC (section, sub-section, etc numbers and titles)
%\listoftables           % inserts LOT (numbers and captions)
%\listoffigures          % inserts LOF (numbers and captions)
%                        %     NOTE: graph chunk must be wrapped with \begin{figure}, 
%                        %  \end{figure}, and \caption{}
%%%%%%%%%%%%%%%%%%%%%%%%%%%%%%%%%%%%%%%%%%%%%%%%%%%%%%%%%%%%%%%%%%%%
% Where everything else goes

\section{How to typeset \textsf{R} code}

If you want to see both the input and output, do this:

\begin{Schunk}
\begin{Sinput}
> runif(10)
\end{Sinput}
\begin{Soutput}
 [1] 0.02614266 0.23460182 0.59832305 0.14789209 0.05683789 0.53273950
 [7] 0.22494575 0.51266456 0.87416583 0.13036323
\end{Soutput}
\end{Schunk}

If you want to see output, but no input, do this:

\begin{Schunk}
\begin{Soutput}
 [1] 0.68058765 0.69535697 0.08438350 0.50416481 0.76245208 0.48381536
 [7] 0.06822326 0.36106798 0.76841479 0.47325312
\end{Soutput}
\end{Schunk}

If you want to see input, but no output, do this:

\begin{Schunk}
\begin{Sinput}
> runif(13)
\end{Sinput}
\end{Schunk}

If you want to run some \textsf{R} code but hide the input/output from the reader then you can do both at the same time:


\bigskip   % leave some empty space (optional)

and you can double-check that it worked later (if you like)

\begin{Schunk}
\begin{Sinput}
> x  # use keep.source=TRUE if you want comments printed
\end{Sinput}
\begin{Soutput}
 [1]  2  3  4  5  6  7  8  9 10 11
\end{Soutput}
\begin{Sinput}
> y
\end{Sinput}
\begin{Soutput}
 [1] 0.47050853 0.46858306 0.06095581 0.22382118 0.13999930 0.66977663
 [7] 0.54393756 0.68595458 0.68244135 0.21761271
\end{Soutput}
\end{Schunk}

If you want to write some \textsf{R} code but not have it evaluated at all then do this:

\begin{Schunk}
\begin{Sinput}
> runif(1e+24)
\end{Sinput}
\end{Schunk}

If you would like to include a figure that's generated completely by \textsf{R} code, then you can do something like the following.

\begin{figure}
\includegraphics{EasySweaveTemplate-007}
\caption{Here is the plot we made}
\end{figure}


Sometimes we would like the output to look like \LaTeX\ output instead of \textsf{R} output.  In that case, do the following.

\begin{Schunk}
\begin{Sinput}
> library(xtable)
> xtable(summary(lm(y ~ x)), caption = "Here is the table we made")
\end{Sinput}
% latex table generated in R 2.13.0 by xtable 1.5-6 package
% Fri May 27 12:11:13 2011
\begin{table}[ht]
\begin{center}
\begin{tabular}{rrrrr}
  \hline
 & Estimate & Std. Error & t value & Pr($>$$|$t$|$) \\ 
  \hline
(Intercept) & 0.2652 & 0.1885 & 1.41 & 0.1970 \\ 
  x & 0.0232 & 0.0265 & 0.88 & 0.4063 \\ 
   \hline
\end{tabular}
\caption{Here is the table we made}
\end{center}
\end{table}\end{Schunk}


\end{document}


