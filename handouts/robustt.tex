% Created 2011-06-06 Mon 11:31
\documentclass[11pt,english]{article}
\usepackage[utf8]{inputenc}
\usepackage[T1]{fontenc}
\usepackage{fixltx2e}
\usepackage{graphicx}
\usepackage{longtable}
\usepackage{float}
\usepackage{wrapfig}
\usepackage{soul}
\usepackage{textcomp}
\usepackage{marvosym}
\usepackage{wasysym}
\usepackage{latexsym}
\usepackage{amssymb}
\usepackage{hyperref}
\tolerance=1000

\usepackage{lmodern}
\renewcommand{\sfdefault}{lmss}
\renewcommand{\ttdefault}{lmtt}

% needed packages
\usepackage{amsmath}
\usepackage{amssymb}
\usepackage{amsthm}
\usepackage{babel}
\usepackage{epsfig}
\usepackage[T1]{fontenc}
\usepackage{fixltx2e}
\usepackage{float}
%\usepackage{floatflt}
\usepackage{graphics}
\usepackage{graphicx}
\usepackage[utf8]{inputenc}
\usepackage{latexsym}
\usepackage{longtable}
\usepackage{makeidx}
\usepackage{marvosym}
\usepackage{multicol}
%\usepackage{pslatex}
\usepackage{rotating}
%\usepackage{showidx}
\usepackage{soul}
\usepackage{srcltx}
\usepackage{stmaryrd}
\usepackage{subfig}
\usepackage{textcomp}
%\usepackage{theorem}
%\usepackage[subfigure]{tocloft}
\usepackage{txfonts}
\usepackage{upgreek}
\usepackage{url}
\usepackage{varioref}
%\usepackage{wasysym}
\usepackage{wrapfig}


% Page setup
\usepackage[paperwidth=8.5in,paperheight=11in]{geometry}
\geometry{verbose,tmargin=0.5in,bmargin=0.5in,lmargin=1in,rmargin=1in}




% PDF settings
%\usepackage[hyperref,x11names]{xcolor}
\usepackage{hyperref}
\hypersetup{pdftitle={STAT 5840: Statistical Computing},
 		pdfauthor={G. Jay Kerns}, 
		linkcolor=Firebrick4, 
		citecolor=black, 
		urlcolor=SteelBlue4}

% Listings setup
%\usepackage{color}
%\usepackage{listings}
%\lstset{basicstyle={\ttfamily},
%	language=R,
%	breaklines=true,
%	breakatwhitespace=true,
%	keywordstyle={\ttfamily},
%	numberstyle = {\ttfamily},
%	morestring=[b]"
%}



%  user defined commands
% special operators
\renewcommand{\P}{\mathrm{I\hspace{-1.5pt}P}}
\newcommand{\E}{\mathrm{I\hspace{-1.5pt}E}}
\renewcommand{\vec}[1]{\mbox{\boldmath$#1$}}

% special symbols
\newcommand{\me}{\mathrm{e}}
\newcommand{\R}{\mathbb{R}}
\newcommand{\diff}{\mathrm{d}}
\newcommand{\ybar}{\overline{y}}
\newcommand{\xbar}{\overline{x}}
\newcommand{\Xbar}{\overline{X}}
\newcommand{\Ybar}{\overline{Y}}





\providecommand{\alert}[1]{\textbf{#1}}

\title{2\#+TITLE:   Investigating Robustness of the Two Sample t-Test}
%\author{G. Jay Kerns}
\date{STAT 5840: Summer 2011}

\begin{document}

\maketitle

\thispagestyle{empty}

In this example we investigate the robustness of the 2 sample \emph{t} statistic.  The setting is $X_{1}$, \ldots{}, $X_{n}$ IID from $f$ and $Y_{1}$, \ldots{}, $Y_{m}$ IID from $g$.  We are interested in seeing how the significance level of the two sample \emph{t}-test changes as we fiddle with the population assumptions.  Copy-paste the \texttt{robustt} function at the command prompt in \texttt{R}.
\begin{verbatim}
# robustt.R

robustt <- function(sigma1, sigma2, Iter = 10000){
  n <- 10; m <- 10         # sample sizes

  frepeated <- function(){
    x <- rnorm(m, sd = sigma1)             # generate data
    y <- rnorm(n, sd = sigma2)  
    tmp <- t.test(x, y, var.equal = TRUE)  # do the test
    tmp$p.value < 0.05                     # did it reject? 
  }

  ptmp <- replicate(Iter, frepeated())     # do it over and over
  p <- mean(ptmp)                          # prop of rejections
  se <- round(sqrt(p*(1-p)/Iter), 4)       # standard error
  
  # report results
  print(paste('sigma1 = ', sigma1, ', sigma2 = ', sigma2, sep = ""))
  print(paste('observed level of test = ', p, ' (', se, ')', sep = ""))
}
\end{verbatim}


\bigskip
\noindent

\textbf{At the command prompt:} we run the simulation to see what happens.  First we try it for the case the variances are identically 1, then we run it again for $\sigma_{1} = 1$, $\sigma_{2} = 10$.

\begin{verbatim}
robustt(1,1)
\end{verbatim}

\begin{verbatim}
 [1] "sigma1 = 1, sigma2 = 1"
 [1] "observed level of test = 0.0502 (0.0022)"
\end{verbatim}


\begin{verbatim}
robustt(1,10)
\end{verbatim}

\begin{verbatim}
 [1] "sigma1 = 1, sigma2 = 10"
 [1] "observed level of test = 0.0659 (0.0025)"
\end{verbatim}



\end{document}