% Created 2011-08-25 Thu 13:07
\documentclass[11pt,english]{article}
\usepackage[utf8]{inputenc}
\usepackage[T1]{fontenc}
\usepackage{fixltx2e}
\usepackage{graphicx}
\usepackage{longtable}
\usepackage{float}
\usepackage{wrapfig}
\usepackage{soul}
\usepackage{textcomp}
\usepackage{marvosym}
\usepackage{wasysym}
\usepackage{latexsym}
\usepackage{amssymb}
\usepackage{hyperref}
\tolerance=1000
\usepackage{color}
\usepackage{listings}

\usepackage{lmodern}
\renewcommand{\sfdefault}{lmss}
\renewcommand{\ttdefault}{lmtt}

% needed packages
\usepackage{amsmath}
\usepackage{amssymb}
\usepackage{amsthm}
\usepackage{babel}
\usepackage{epsfig}
\usepackage[T1]{fontenc}
\usepackage{fixltx2e}
\usepackage{float}
%\usepackage{floatflt}
\usepackage{graphics}
\usepackage{graphicx}
\usepackage[utf8]{inputenc}
\usepackage{latexsym}
\usepackage{longtable}
\usepackage{makeidx}
\usepackage{marvosym}
\usepackage{multicol}
%\usepackage{pslatex}
\usepackage{rotating}
%\usepackage{showidx}
\usepackage{soul}
\usepackage{srcltx}
\usepackage{stmaryrd}
\usepackage{subfig}
\usepackage{textcomp}
%\usepackage{theorem}
%\usepackage[subfigure]{tocloft}
\usepackage{txfonts}
\usepackage{upgreek}
\usepackage{url}
\usepackage{varioref}
%\usepackage{wasysym}
\usepackage{wrapfig}


% Page setup
\usepackage[paperwidth=8.5in,paperheight=11in]{geometry}
\geometry{verbose,tmargin=0.5in,bmargin=0.5in,lmargin=1in,rmargin=1in}




% PDF settings
%\usepackage[hyperref,x11names]{xcolor}
\usepackage{hyperref}
\hypersetup{pdftitle={STAT 5840: Statistical Computing},
 		pdfauthor={G. Jay Kerns}, 
		linkcolor=Firebrick4, 
		citecolor=black, 
		urlcolor=SteelBlue4}

% Listings setup
%\usepackage{color}
%\usepackage{listings}
%\lstset{basicstyle={\ttfamily},
%	language=R,
%	breaklines=true,
%	breakatwhitespace=true,
%	keywordstyle={\ttfamily},
%	numberstyle = {\ttfamily},
%	morestring=[b]"
%}



%  user defined commands
% special operators
\renewcommand{\P}{\mathrm{I\hspace{-1.5pt}P}}
\newcommand{\E}{\mathrm{I\hspace{-1.5pt}E}}
\renewcommand{\vec}[1]{\mbox{\boldmath$#1$}}

% special symbols
\newcommand{\me}{\mathrm{e}}
\newcommand{\R}{\mathbb{R}}
\newcommand{\diff}{\mathrm{d}}
\newcommand{\ybar}{\overline{y}}
\newcommand{\xbar}{\overline{x}}
\newcommand{\Xbar}{\overline{X}}
\newcommand{\Ybar}{\overline{Y}}





\providecommand{\alert}[1]{\textbf{#1}}

\title{Assignment: Importance Sampling and Antithetic Variables}
\author{STAT 5840 Summer 2011}
\date{}

\begin{document}

\maketitle


\section*{Questions}
\label{sec-1}



\begin{enumerate}
\item Obtain a Monte Carlo estimate $\hat{\theta}$ of 
   \[
   \int_{1}^{\infty} \frac{x^{3}}{\sqrt{2\pi}}\me^{-x^2/2} \diff x
   \]
   by importance sampling,  and also estimate the variance of $\hat{\theta}$.  \textbf{Note:} there are \emph{many} ways to do this problem, and I'm only looking for one (1) way.  Surprise me.
\item Compute a Monte Carlo estimate $\hat{\theta}$ of 
   \[
   \theta = \int_{0}^{2} \me ^{-x}\,\diff x
   \]
   by sampling from a Unif(0,2) distribution, and also estimate the variance of $\hat{\theta}$. Find another Monte Carlo estimator $\hat{\theta}^{\ast}$ by sampling from an Exp(1) distribution.  Which of the variances (of $\hat{\theta}$ and $\hat{\theta}^{\ast}$) is smaller, and why?
\item Use Monte Carlo integration with antithetic variables to estimate
   \[
   \theta = \int_{0}^{1} \frac{\me ^{-x}}{1 + x^{2}}\,\diff x,
   \]
   and find the approximate reduction in variance as a percentage of the variance without variance reduction (this means you also need to estimate $\theta$ the naive way and compare variances).
\end{enumerate}

\textbf{Hints:}


\begin{enumerate}
\item Look for an importance function which is easy to simulate and has heavier tails than $f$ (but still has finite variance).  You can also manipulate the integral beforehand with Calculus tricks which may give you something easier/better.  For hints how to set up your program, look at \texttt{explognormal.R}.
\item Not really much to say here.  Simulate some variates, compute some values, and find the \texttt{mean}.  For the variance you are looking at $\hat{\theta}(1 - \hat{\theta})/n$.
\item Take a look at the range of integration.  Do you see the natural choice of variates to simulate?  Now, remember from class: what's the easiest way to introduce correlation with variates like those?
\end{enumerate}

\end{document}