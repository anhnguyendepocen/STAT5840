% Created 2011-08-25 Thu 13:05
\documentclass[11pt,english]{article}
\usepackage[utf8]{inputenc}
\usepackage[T1]{fontenc}
\usepackage{fixltx2e}
\usepackage{graphicx}
\usepackage{longtable}
\usepackage{float}
\usepackage{wrapfig}
\usepackage{soul}
\usepackage{textcomp}
\usepackage{marvosym}
\usepackage{wasysym}
\usepackage{latexsym}
\usepackage{amssymb}
\usepackage{hyperref}
\tolerance=1000
\usepackage{color}
\usepackage{listings}

\usepackage{lmodern}
\renewcommand{\sfdefault}{lmss}
\renewcommand{\ttdefault}{lmtt}

% needed packages
\usepackage{amsmath}
\usepackage{amssymb}
\usepackage{amsthm}
\usepackage{babel}
\usepackage{epsfig}
\usepackage[T1]{fontenc}
\usepackage{fixltx2e}
\usepackage{float}
%\usepackage{floatflt}
\usepackage{graphics}
\usepackage{graphicx}
\usepackage[utf8]{inputenc}
\usepackage{latexsym}
\usepackage{longtable}
\usepackage{makeidx}
\usepackage{marvosym}
\usepackage{multicol}
%\usepackage{pslatex}
\usepackage{rotating}
%\usepackage{showidx}
\usepackage{soul}
\usepackage{srcltx}
\usepackage{stmaryrd}
\usepackage{subfig}
\usepackage{textcomp}
%\usepackage{theorem}
%\usepackage[subfigure]{tocloft}
\usepackage{txfonts}
\usepackage{upgreek}
\usepackage{url}
\usepackage{varioref}
%\usepackage{wasysym}
\usepackage{wrapfig}


% Page setup
\usepackage[paperwidth=8.5in,paperheight=11in]{geometry}
\geometry{verbose,tmargin=0.5in,bmargin=0.5in,lmargin=1in,rmargin=1in}




% PDF settings
%\usepackage[hyperref,x11names]{xcolor}
\usepackage{hyperref}
\hypersetup{pdftitle={STAT 5840: Statistical Computing},
 		pdfauthor={G. Jay Kerns}, 
		linkcolor=Firebrick4, 
		citecolor=black, 
		urlcolor=SteelBlue4}

% Listings setup
%\usepackage{color}
%\usepackage{listings}
%\lstset{basicstyle={\ttfamily},
%	language=R,
%	breaklines=true,
%	breakatwhitespace=true,
%	keywordstyle={\ttfamily},
%	numberstyle = {\ttfamily},
%	morestring=[b]"
%}



%  user defined commands
% special operators
\renewcommand{\P}{\mathrm{I\hspace{-1.5pt}P}}
\newcommand{\E}{\mathrm{I\hspace{-1.5pt}E}}
\renewcommand{\vec}[1]{\mbox{\boldmath$#1$}}

% special symbols
\newcommand{\me}{\mathrm{e}}
\newcommand{\R}{\mathbb{R}}
\newcommand{\diff}{\mathrm{d}}
\newcommand{\ybar}{\overline{y}}
\newcommand{\xbar}{\overline{x}}
\newcommand{\Xbar}{\overline{X}}
\newcommand{\Ybar}{\overline{Y}}





\providecommand{\alert}[1]{\textbf{#1}}

\title{Comparing the Risks of Three Classical Estimators}
%\author{G. Jay Kerns}
\date{STAT 5840: Summer 2011}

\begin{document}

\maketitle

\thispagestyle{empty}

Here we go to compare risks of three classical estimators: the sample mean, sample median, and trimmed sample mean.  Our underlying population will be $\mathrm{Laplace}(\mu,1)$ and the loss will be squared error loss.

\section*{The Script}
\label{sec-1}



\begin{verbatim}
# compareestimators.R

set.seed(1)
Iter <- 1000        # number of iterations (should be multiple of 100)
n <- 20             # size of random sample
tr <- 0.05          # proportion of obs trimmed from end of sample

slmn <- 0           # the "sum of the loss for the mean"=0
slmd <- 0           # the "sum of the loss for the median"=0
sltm <- 0           # the "sum of the loss for the trimmed mean"=0

sl2mn <- 0          # the "sum of the squared loss for the mean"=0
sl2md <- 0          # the "sum of the squared loss for the median"=0
sl2tm <- 0          # the "sum of the squared loss for the trimmed mean"=0

mu <- 0             # location parameter assumed to be zero

for (i in seq.int(Iter)){
  # simulate n Laplace r.v.s
  x <- log(runif(n)/runif(n))  #n-vector of Laplace(0,1)s
                             
  loss1 <- (mean(x) - mu)^2    # compute losses
  loss2 <- (median(x) - mu)^2
  loss3 <- (mean(x, trim = tr) - mu)^2

  slmn <- slmn + loss1         # accumulate loss for mean
  slmd <- slmd + loss2         # accumulate loss for median
  sltm <- sltm + loss3         # accumulate loss for trimmed mean

  sl2mn <- sl2mn + loss1^2     # accumulate loss^2 for mean
  sl2md <- sl2md + loss2^2     # accumulate loss^2 for median
  sl2tm <- sl2tm + loss3^2     # accumulate loss^2 for median
}

# Now compute Monte Carlo risks

r1 <- round(slmn/Iter, 4)              # average loss
r2 <- round(slmd/Iter, 4)
r3 <- round(sltm/Iter, 4)

# Standard error of Loss
se1 <- round(sqrt((sl2mn - slmn^2/Iter)/Iter^2), 4)
se2 <- round(sqrt((sl2md - slmd^2/Iter)/Iter^2), 4)
se3 <- round(sqrt((sl2tm - sltm^2/Iter)/Iter^2), 4)
\end{verbatim}
\section*{At the command prompt}
\label{sec-2}

We copy-paste the above script and let it run. To see what happened we may do the following.


\begin{verbatim}
# Now print the results
paste("Mean:         ", r1, " (", se1, ")", sep="")
paste("Median:       ", r2, " (", se2, ")", sep="")
paste("Trimmed Mean: ", r3, " (", se3, ")", sep="")
\end{verbatim}




\begin{verbatim}
 [1] "Mean:         0.0994 (0.0045)"
 [1] "Median:       0.0685 (0.0037)"
 [1] "Trimmed Mean: 0.0864 (0.0039)"
\end{verbatim}

\end{document}