% Created 2011-08-26 Fri 15:58
\documentclass[11pt,english]{article}

\usepackage{color}
\usepackage{listings}

\usepackage{lmodern}
\renewcommand{\sfdefault}{lmss}
\renewcommand{\ttdefault}{lmtt}

% needed packages
\usepackage{amsmath}
\usepackage{amssymb}
\usepackage{amsthm}
\usepackage{babel}
\usepackage{epsfig}
\usepackage[T1]{fontenc}
\usepackage{fixltx2e}
\usepackage{float}
%\usepackage{floatflt}
\usepackage{graphics}
\usepackage{graphicx}
\usepackage[utf8]{inputenc}
\usepackage{latexsym}
\usepackage{longtable}
\usepackage{makeidx}
\usepackage{marvosym}
\usepackage{multicol}
%\usepackage{pslatex}
\usepackage{rotating}
%\usepackage{showidx}
\usepackage{soul}
\usepackage{srcltx}
\usepackage{stmaryrd}
\usepackage{subfig}
\usepackage{textcomp}
%\usepackage{theorem}
%\usepackage[subfigure]{tocloft}
\usepackage{txfonts}
\usepackage{upgreek}
\usepackage{url}
\usepackage{varioref}
%\usepackage{wasysym}
\usepackage{wrapfig}


% Page setup
\usepackage[paperwidth=8.5in,paperheight=11in]{geometry}
\geometry{verbose,tmargin=0.5in,bmargin=0.5in,lmargin=1in,rmargin=1in}




% PDF settings
%\usepackage[hyperref,x11names]{xcolor}
\usepackage{hyperref}
\hypersetup{pdftitle={STAT 5840: Statistical Computing},
 		pdfauthor={G. Jay Kerns}, 
		linkcolor=Firebrick4, 
		citecolor=black, 
		urlcolor=SteelBlue4}

% Listings setup
%\usepackage{color}
%\usepackage{listings}
%\lstset{basicstyle={\ttfamily},
%	language=R,
%	breaklines=true,
%	breakatwhitespace=true,
%	keywordstyle={\ttfamily},
%	numberstyle = {\ttfamily},
%	morestring=[b]"
%}



%  user defined commands
% special operators
\renewcommand{\P}{\mathrm{I\hspace{-1.5pt}P}}
\newcommand{\E}{\mathrm{I\hspace{-1.5pt}E}}
\renewcommand{\vec}[1]{\mbox{\boldmath$#1$}}

% special symbols
\newcommand{\me}{\mathrm{e}}
\newcommand{\R}{\mathbb{R}}
\newcommand{\diff}{\mathrm{d}}
\newcommand{\ybar}{\overline{y}}
\newcommand{\xbar}{\overline{x}}
\newcommand{\Xbar}{\overline{X}}
\newcommand{\Ybar}{\overline{Y}}





\providecommand{\alert}[1]{\textbf{#1}}

\title{Bootstrapping: still the long way}
%\author{}
\date{STAT 5840: Summer 2011}

\begin{document}

\maketitle

\thispagestyle{empty}

\section*{Bootstrapping the standard error of the sample median}
\label{sec-1}

This program illustrates the Bootstap procedure for estimating the standard error of the sample median.  The data are the built-in data vector \texttt{islands}, which represents the areas in thousands of square miles of the landmasses which exceed 10,000 square miles.




\begin{verbatim}
# bootmed.R
n <- length(islands)     
Iter <- 200
medstar <- rep(0, times = Iter)
for(i in seq.int(Iter)){
  boot.samp <- sample(islands, size = n, replace = TRUE)
  medstar[i] <- median(boot.samp)
}
\end{verbatim}




After the copy-paste we can check the results with the following.

\begin{verbatim}
mean(medstar)
mean(islands)
median(medstar)
median(islands)
sd(medstar)
\end{verbatim}




\begin{verbatim}
 [1] 40.04
 [1] 1252.729
 [1] 40
 [1] 41
 [1] 11.44556
\end{verbatim}


We should not trust our estimate of the standard error of the median, here, because we can see that the data are substantially skewed.




\begin{figure}[h!]
\centering
\includegraphics[width=6in, height=6in,]{img/bootmed.pdf}
\caption{\label{fig:yplot}Histogram of bootstrap replicates for the sample median}
\end{figure}

\end{document}