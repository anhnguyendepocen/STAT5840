% Created 2011-08-26 Fri 16:03
\documentclass[11pt,english]{article}

\usepackage{color}
\usepackage{listings}

\usepackage{lmodern}
\renewcommand{\sfdefault}{lmss}
\renewcommand{\ttdefault}{lmtt}

% needed packages
\usepackage{amsmath}
\usepackage{amssymb}
\usepackage{amsthm}
\usepackage{babel}
\usepackage{epsfig}
\usepackage[T1]{fontenc}
\usepackage{fixltx2e}
\usepackage{float}
%\usepackage{floatflt}
\usepackage{graphics}
\usepackage{graphicx}
\usepackage[utf8]{inputenc}
\usepackage{latexsym}
\usepackage{longtable}
\usepackage{makeidx}
\usepackage{marvosym}
\usepackage{multicol}
%\usepackage{pslatex}
\usepackage{rotating}
%\usepackage{showidx}
\usepackage{soul}
\usepackage{srcltx}
\usepackage{stmaryrd}
\usepackage{subfig}
\usepackage{textcomp}
%\usepackage{theorem}
%\usepackage[subfigure]{tocloft}
\usepackage{txfonts}
\usepackage{upgreek}
\usepackage{url}
\usepackage{varioref}
%\usepackage{wasysym}
\usepackage{wrapfig}


% Page setup
\usepackage[paperwidth=8.5in,paperheight=11in]{geometry}
\geometry{verbose,tmargin=0.5in,bmargin=0.5in,lmargin=1in,rmargin=1in}




% PDF settings
%\usepackage[hyperref,x11names]{xcolor}
\usepackage{hyperref}
\hypersetup{pdftitle={STAT 5840: Statistical Computing},
 		pdfauthor={G. Jay Kerns}, 
		linkcolor=Firebrick4, 
		citecolor=black, 
		urlcolor=SteelBlue4}

% Listings setup
%\usepackage{color}
%\usepackage{listings}
%\lstset{basicstyle={\ttfamily},
%	language=R,
%	breaklines=true,
%	breakatwhitespace=true,
%	keywordstyle={\ttfamily},
%	numberstyle = {\ttfamily},
%	morestring=[b]"
%}



%  user defined commands
% special operators
\renewcommand{\P}{\mathrm{I\hspace{-1.5pt}P}}
\newcommand{\E}{\mathrm{I\hspace{-1.5pt}E}}
\renewcommand{\vec}[1]{\mbox{\boldmath$#1$}}

% special symbols
\newcommand{\me}{\mathrm{e}}
\newcommand{\R}{\mathbb{R}}
\newcommand{\diff}{\mathrm{d}}
\newcommand{\ybar}{\overline{y}}
\newcommand{\xbar}{\overline{x}}
\newcommand{\Xbar}{\overline{X}}
\newcommand{\Ybar}{\overline{Y}}





\providecommand{\alert}[1]{\textbf{#1}}

\title{The Jackknife}
%\author{}
\date{STAT 5840: Summer 2011}

\begin{document}

\maketitle

\thispagestyle{empty}

\section*{Estimating bias of a ratio estimator with the jackknife\footnote{Adapted from \emph{Statistical Computing with R} by Maria Rizzo (2008). }}
\label{sec-1}

The \texttt{patch} data in the \texttt{bootstrap} package (you may need to install the package if you do not already have it) has data about bloodstream measurements for eight subjects in a study.  The data look like this:

\begin{verbatim}
library(bootstrap)
data(patch)
head(patch)
\end{verbatim}




\begin{verbatim}
   subject placebo oldpatch newpatch    z     y
 1       1    9243    17649    16449 8406 -1200
 2       2    9671    12013    14614 2342  2601
 3       3   11792    19979    17274 8187 -2705
 4       4   13357    21816    23798 8459  1982
 5       5    9055    13850    12560 4795 -1290
 6       6    6290     9806    10157 3516   351
\end{verbatim}


The parameter of interest is
\[
\theta = \frac{\E[\mathrm{new}] - \E[\mathrm{old}]}{\E[\mathrm{old}] - \E[\mathrm{placebo}]},
\]
which we can estimate with the statistic \(\hat{\theta} = \overline{Y}/\overline{Z}\).  The following uses the jackknife to estimate the bias of $\hat{\theta}$.


\begin{verbatim}
# jackknife1.R
n <- nrow(patch)
y <- patch$y
z <- patch$z
theta.hat <- mean(y)/mean(z)  # original value of statistic

theta.jack <- numeric(n)
for (i in 1:n){
  theta.jack[i] <- mean(y[-i]) / mean(z[-i])
}
bias <- (n - 1) * (mean(theta.jack) - theta.hat)
\end{verbatim}




After the copy-paste of the above we may look at our results with the following.

\begin{verbatim}
theta.hat
bias
\end{verbatim}




\begin{verbatim}
 \protect\footnotemark[1] -0.0713061
 \footnotemark[1] 0.008002488
\end{verbatim}

\end{document}