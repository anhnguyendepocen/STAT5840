% Created 2011-06-10 Fri 11:22
\documentclass[11pt,english]{article}
\usepackage[utf8]{inputenc}
\usepackage[T1]{fontenc}
\usepackage{fixltx2e}
\usepackage{graphicx}
\usepackage{longtable}
\usepackage{float}
\usepackage{wrapfig}
\usepackage{soul}
\usepackage{textcomp}
\usepackage{marvosym}
\usepackage{wasysym}
\usepackage{latexsym}
\usepackage{amssymb}
\usepackage{hyperref}
\tolerance=1000

\usepackage{lmodern}
\renewcommand{\sfdefault}{lmss}
\renewcommand{\ttdefault}{lmtt}

% needed packages
\usepackage{amsmath}
\usepackage{amssymb}
\usepackage{amsthm}
\usepackage{babel}
\usepackage{epsfig}
\usepackage[T1]{fontenc}
\usepackage{fixltx2e}
\usepackage{float}
%\usepackage{floatflt}
\usepackage{graphics}
\usepackage{graphicx}
\usepackage[utf8]{inputenc}
\usepackage{latexsym}
\usepackage{longtable}
\usepackage{makeidx}
\usepackage{marvosym}
\usepackage{multicol}
%\usepackage{pslatex}
\usepackage{rotating}
%\usepackage{showidx}
\usepackage{soul}
\usepackage{srcltx}
\usepackage{stmaryrd}
\usepackage{subfig}
\usepackage{textcomp}
%\usepackage{theorem}
%\usepackage[subfigure]{tocloft}
\usepackage{txfonts}
\usepackage{upgreek}
\usepackage{url}
\usepackage{varioref}
%\usepackage{wasysym}
\usepackage{wrapfig}


% Page setup
\usepackage[paperwidth=8.5in,paperheight=11in]{geometry}
\geometry{verbose,tmargin=0.5in,bmargin=0.5in,lmargin=1in,rmargin=1in}




% PDF settings
%\usepackage[hyperref,x11names]{xcolor}
\usepackage{hyperref}
\hypersetup{pdftitle={STAT 5840: Statistical Computing},
 		pdfauthor={G. Jay Kerns}, 
		linkcolor=Firebrick4, 
		citecolor=black, 
		urlcolor=SteelBlue4}

% Listings setup
%\usepackage{color}
%\usepackage{listings}
%\lstset{basicstyle={\ttfamily},
%	language=R,
%	breaklines=true,
%	breakatwhitespace=true,
%	keywordstyle={\ttfamily},
%	numberstyle = {\ttfamily},
%	morestring=[b]"
%}



%  user defined commands
% special operators
\renewcommand{\P}{\mathrm{I\hspace{-1.5pt}P}}
\newcommand{\E}{\mathrm{I\hspace{-1.5pt}E}}
\renewcommand{\vec}[1]{\mbox{\boldmath$#1$}}

% special symbols
\newcommand{\me}{\mathrm{e}}
\newcommand{\R}{\mathbb{R}}
\newcommand{\diff}{\mathrm{d}}
\newcommand{\ybar}{\overline{y}}
\newcommand{\xbar}{\overline{x}}
\newcommand{\Xbar}{\overline{X}}
\newcommand{\Ybar}{\overline{Y}}





\providecommand{\alert}[1]{\textbf{#1}}

\title{Assignment: Evaluating Tests via Monte Carlo Methods}
\author{}
\date{}

\begin{document}

\maketitle


\section*{Setup}
\label{sec-1}


Suppose $X_{1}$, $X_{2}$, \ldots{}, $X_{n}$ are a random sample from a distribution $f$ with mean $\mu$.  The standard test of the null hypothesis $H_{0}: \mu = 0$ is based on the statistic
\[
T = \frac{\Xbar}{S/\sqrt{n}}.
\]
At level $\alpha = 0.05$, one rejects $H_{0}$ if $|T| > t_{\alpha/2}$, where $t_{\alpha/2}$ is the percentile of a \texttt{t(df = n-1)} distribution with upper tail probability 0.025.  For this exercise, assume $n = 10$.


\begin{description}
\item[KNOW:] An implicit assumption for the above test is that the underlying population is Normal. What happens if that assumption is violated?
\end{description}
\section*{Questions.}
\label{sec-2}


\begin{itemize}
\item Using Monte Carlo simulation, approximate the level of this test when $f$ is
\begin{enumerate}
\item N(0,1),
\item Chi-square with 4 \texttt{df},
\item Uniform(0,1), and
\item Standard Logistic, Logis(0,1).
\end{enumerate}
\item For each population distribution, give a) the estimate of the level, b) a standard error for your estimate, and c) a paragraph stating what you have found.
\item After the four parts are finished, write a report collecting and summarizing your findings.  What have you learned about the robustness of the \emph{t}-test to departures from the normality assumption?
\end{itemize}

\textbf{Hints:}

\begin{itemize}
\item Download \texttt{robust.R} and modify it to suit your needs by changing the simulated data and the \texttt{t.test} command.
\end{itemize}

\end{document}