% Created 2011-05-18 Wed 12:10
\documentclass[11pt,english]{article}
\usepackage[utf8]{inputenc}
\usepackage[T1]{fontenc}
\usepackage{fixltx2e}
\usepackage{graphicx}
\usepackage{longtable}
\usepackage{float}
\usepackage{wrapfig}
\usepackage{soul}
\usepackage{textcomp}
\usepackage{marvosym}
\usepackage{wasysym}
\usepackage{latexsym}
\usepackage{amssymb}
\usepackage{hyperref}
\tolerance=1000

\usepackage{lmodern}
\renewcommand{\sfdefault}{lmss}
\renewcommand{\ttdefault}{lmtt}

% needed packages
\usepackage{amsmath}
\usepackage{amssymb}
\usepackage{amsthm}
\usepackage{babel}
\usepackage{epsfig}
\usepackage[T1]{fontenc}
\usepackage{fixltx2e}
\usepackage{float}
%\usepackage{floatflt}
\usepackage{graphics}
\usepackage{graphicx}
\usepackage[utf8]{inputenc}
\usepackage{latexsym}
\usepackage{longtable}
\usepackage{makeidx}
\usepackage{marvosym}
\usepackage{multicol}
%\usepackage{pslatex}
\usepackage{rotating}
%\usepackage{showidx}
\usepackage{soul}
\usepackage{srcltx}
\usepackage{stmaryrd}
\usepackage{subfig}
\usepackage{textcomp}
%\usepackage{theorem}
%\usepackage[subfigure]{tocloft}
\usepackage{txfonts}
\usepackage{upgreek}
\usepackage{url}
\usepackage{varioref}
%\usepackage{wasysym}
\usepackage{wrapfig}


% Page setup
\usepackage[paperwidth=8.5in,paperheight=11in]{geometry}
\geometry{verbose,tmargin=0.5in,bmargin=0.5in,lmargin=1in,rmargin=1in}




% PDF settings
%\usepackage[hyperref,x11names]{xcolor}
\usepackage{hyperref}
\hypersetup{pdftitle={STAT 5840: Statistical Computing},
 		pdfauthor={G. Jay Kerns}, 
		linkcolor=Firebrick4, 
		citecolor=black, 
		urlcolor=SteelBlue4}

% Listings setup
%\usepackage{color}
%\usepackage{listings}
%\lstset{basicstyle={\ttfamily},
%	language=R,
%	breaklines=true,
%	breakatwhitespace=true,
%	keywordstyle={\ttfamily},
%	numberstyle = {\ttfamily},
%	morestring=[b]"
%}



%  user defined commands
% special operators
\renewcommand{\P}{\mathrm{I\hspace{-1.5pt}P}}
\newcommand{\E}{\mathrm{I\hspace{-1.5pt}E}}
\renewcommand{\vec}[1]{\mbox{\boldmath$#1$}}

% special symbols
\newcommand{\me}{\mathrm{e}}
\newcommand{\R}{\mathbb{R}}
\newcommand{\diff}{\mathrm{d}}
\newcommand{\ybar}{\overline{y}}
\newcommand{\xbar}{\overline{x}}
\newcommand{\Xbar}{\overline{X}}
\newcommand{\Ybar}{\overline{Y}}





\providecommand{\alert}[1]{\textbf{#1}}

\title{Accept-Reject Algorithm}
\author{}
\date{}

\begin{document}

\maketitle


\section*{Simulating Beta(2.5, 4.5) Random Variables: Which method is better?}
\label{sec-1}

\noindent
We wish to simulate Beta(2.5,4.5) random variables with target density
\[
f(x) \propto x^{1.5}(1-x)^{3.5}.
\]
To do this we will use the Accept/Reject Algorithm with two different instrumental densities.


\begin{description}
\item[Method I:] Use uniform RV’s with instrumental density
   \[
   g_{1}(x) = 1,\quad 0 < x < 1.
   \]
   \textbf{Notes:}
\begin{itemize}
\item you may use \(M \approx 0.0472\).
\item to simulate from \(g_{1}\) use the command \texttt{x <- runif(1)}.
\end{itemize}
\item[Method II:] Use the average of two uniform RV’s with instrumental density
   \[
   g_{2}(x) = 2 - |4x - 2|,\quad 0 < x <1.
   \]
   \textbf{Notes:}
\begin{itemize}
\item you may use \(M \approx 0.0554\).
\item the \texttt{R} function for absolute value is \texttt{abs}.
\item to simulate from \(g_{2}\) use the command \texttt{x <- mean(runif(2))}.
\end{itemize}
\end{description}
\section*{Question: Which method is better?}
\label{sec-2}


\begin{enumerate}
\item Write a program that simulates 10,000 Beta(2.5,4.5) random variables using both methods.
\item Compare the acceptance rates and decide which one is the better algorithm.
\item Graph histograms of your results.
\item Write a report that includes your program, your histograms, and a paragraph that summarizes your findings.
\end{enumerate}

\textbf{Hints:}

\begin{itemize}
\item Download \texttt{rand\_norm.R} and modify it to suit your needs by changing the target and instrumental densities.  You could call the new one \texttt{rand\_beta.R}.
\item It may help to write two separate programs, \texttt{rand\_beta1.R} and \texttt{rand\_beta2.R}.
\end{itemize}

\end{document}