% Created 2011-08-25 Thu 13:08
\documentclass[11pt,english]{article}
\usepackage[utf8]{inputenc}
\usepackage[T1]{fontenc}
\usepackage{fixltx2e}
\usepackage{graphicx}
\usepackage{longtable}
\usepackage{float}
\usepackage{wrapfig}
\usepackage{soul}
\usepackage{textcomp}
\usepackage{marvosym}
\usepackage{wasysym}
\usepackage{latexsym}
\usepackage{amssymb}
\usepackage{hyperref}
\tolerance=1000
\usepackage{color}
\usepackage{listings}

\usepackage{lmodern}
\renewcommand{\sfdefault}{lmss}
\renewcommand{\ttdefault}{lmtt}

% needed packages
\usepackage{amsmath}
\usepackage{amssymb}
\usepackage{amsthm}
\usepackage{babel}
\usepackage{epsfig}
\usepackage[T1]{fontenc}
\usepackage{fixltx2e}
\usepackage{float}
%\usepackage{floatflt}
\usepackage{graphics}
\usepackage{graphicx}
\usepackage[utf8]{inputenc}
\usepackage{latexsym}
\usepackage{longtable}
\usepackage{makeidx}
\usepackage{marvosym}
\usepackage{multicol}
%\usepackage{pslatex}
\usepackage{rotating}
%\usepackage{showidx}
\usepackage{soul}
\usepackage{srcltx}
\usepackage{stmaryrd}
\usepackage{subfig}
\usepackage{textcomp}
%\usepackage{theorem}
%\usepackage[subfigure]{tocloft}
\usepackage{txfonts}
\usepackage{upgreek}
\usepackage{url}
\usepackage{varioref}
%\usepackage{wasysym}
\usepackage{wrapfig}


% Page setup
\usepackage[paperwidth=8.5in,paperheight=11in]{geometry}
\geometry{verbose,tmargin=0.5in,bmargin=0.5in,lmargin=1in,rmargin=1in}




% PDF settings
%\usepackage[hyperref,x11names]{xcolor}
\usepackage{hyperref}
\hypersetup{pdftitle={STAT 5840: Statistical Computing},
 		pdfauthor={G. Jay Kerns}, 
		linkcolor=Firebrick4, 
		citecolor=black, 
		urlcolor=SteelBlue4}

% Listings setup
%\usepackage{color}
%\usepackage{listings}
%\lstset{basicstyle={\ttfamily},
%	language=R,
%	breaklines=true,
%	breakatwhitespace=true,
%	keywordstyle={\ttfamily},
%	numberstyle = {\ttfamily},
%	morestring=[b]"
%}



%  user defined commands
% special operators
\renewcommand{\P}{\mathrm{I\hspace{-1.5pt}P}}
\newcommand{\E}{\mathrm{I\hspace{-1.5pt}E}}
\renewcommand{\vec}[1]{\mbox{\boldmath$#1$}}

% special symbols
\newcommand{\me}{\mathrm{e}}
\newcommand{\R}{\mathbb{R}}
\newcommand{\diff}{\mathrm{d}}
\newcommand{\ybar}{\overline{y}}
\newcommand{\xbar}{\overline{x}}
\newcommand{\Xbar}{\overline{X}}
\newcommand{\Ybar}{\overline{Y}}





\providecommand{\alert}[1]{\textbf{#1}}

\title{Accelerating Monte Carlo Integration}
%\author{}
\date{STAT 5840: Summer 2011}

\begin{document}

\maketitle

\thispagestyle{empty}

\section*{Evaluating the James-Stein Estimator}
\label{sec-1}

This program illustrates risk calculations for the James-Stein estimator using Monte Carlo integration.  In this example we have 
\[
X = \left( X_{1},X_{2},\ldots,X_{p}),\mbox{ with }X_{i}\sim N(\theta,1),\ i=1,\ldots 5. 
\]
and we we will take $p = 5$ and $\theta$ ranging from $0 < \theta < 10$. We are using the James-Stein Estimator:
\[
\delta_{i}^{\mathrm{JS}}(X) = \left(1 - \frac{a}{|X|^{2}} \right)^{+} X_{i}, \quad 0 < a < 2(p - 2),
\]
 We do it in three parts:

\begin{enumerate}
\item Naive way: generate a bunch of normals (\texttt{Iter} = 2000) like crazy and see what happens.
\item Clever way: only generate one (1) master set of normals and reuse those for all values of $\theta$.
\item Do part 2) over and over for different choices of $a$  Use the same set of normals for all $\theta$ and for each $a$.
\end{enumerate}
\subsection*{Here's the first part.}
\label{sec-1-1}



\begin{verbatim}
# jamesstein.R
set.seed(1)
p <- 5             # how many populations
Iter <- 2000       # sample size
n <- 100           # number of thetas for the grid
theta <- seq(from = 0, to = 10, length.out = n)  # thetas on the grid
a <- 3             # J-S parameter, 0 < a < 2(p-2)
risk <- rep(0, times = n)

# Naive simulation
for (i in seq.int(n)){
  x <- matrix(rnorm(p*Iter, mean = theta[i]), nrow = p)
  nx <- colSums(x^2)
  js1 <- max(c(0, 1 - a/nx)) * x[1, ]
  js2 <- max(c(0, 1 - a/nx)) * x[2, ]
  js3 <- max(c(0, 1 - a/nx)) * x[3, ]
  js4 <- max(c(0, 1 - a/nx)) * x[4, ]
  js5 <- max(c(0, 1 - a/nx)) * x[5, ]
  risk[i] <- mean((js1 - theta[i])^2 + (js1 - theta[i])^2 + 
                  (js3 - theta[i])^2 + (js4 - theta[i])^2 + 
                  (js5 - theta[i])^2)
}
\end{verbatim}
\subsection*{Here's the second part.}
\label{sec-1-2}



\begin{verbatim}
# Instead use the same sequence of variates
risk2 <- rep(0, times = n)
xstay <- matrix(rnorm(p*Iter), nrow = p)  # generate one set of data

for (i in seq.int(n)){
  x <- xstay + theta[i]
  nx <- colSums(x^2)
  js1 <- max(c(0, 1 - a/nx)) * x[1, ]
  js2 <- max(c(0, 1 - a/nx)) * x[2, ]
  js3 <- max(c(0, 1 - a/nx)) * x[3, ]
  js4 <- max(c(0, 1 - a/nx)) * x[4, ]
  js5 <- max(c(0, 1 - a/nx)) * x[5, ]
  risk2[i] <- mean((js1 - theta[i])^2 + (js1 - theta[i])^2 + 
                   (js3 - theta[i])^2 + (js4 - theta[i])^2 + 
                   (js5 - theta[i])^2)
}
\end{verbatim}
\subsection*{Here's the third part.}
\label{sec-1-3}



\begin{verbatim}
# Now evaluate for different choices of a
a <- 1:5
risk3 <- matrix(rep(0, times = n*length(a)), nrow  = length(a))
th <- matrix(rnorm(p*Iter), nrow = p)

for (k in seq_along(a)){
  for (i in seq.int(n)){
    x <- th + theta[i]
    nx <- colSums(x^2)
    js1 <- max(c(0, 1 - a[k]/nx)) * x[1, ]
    js2 <- max(c(0, 1 - a[k]/nx)) * x[2, ]
    js3 <- max(c(0, 1 - a[k]/nx)) * x[3, ]
    js4 <- max(c(0, 1 - a[k]/nx)) * x[4, ]
    js5 <- max(c(0, 1 - a[k]/nx)) * x[5, ]
    risk3[k, i] <- mean((js1 - theta[i])^2 + (js1 - theta[i])^2 + 
                        (js3 - theta[i])^2 + (js4 - theta[i])^2 + 
                        (js5 - theta[i])^2)
  }
}
\end{verbatim}






\begin{figure}[h!]
\centering
\includegraphics[width=4in, height=4in,]{img/jamesstein.pdf}
\caption{\label{fig:yplot}Plots for James-Stein risk estimation, naive method (wiggly) and reuse method (smooth) when $a = 3$}
\end{figure}




\begin{figure}[h!]
\centering
\includegraphics[width=4in, height=4in,]{img/jamesstein2.pdf}
\caption{\label{fig:yplot}Plots for James-Stein risk estimation, as $a$ ranges from $a = 1$ (skinny) to $a = 5$ (wide)}
\end{figure}

\end{document}